\documentclass[english]{article}%{llncs}
\usepackage[english]{babel}
\usepackage[utf8]{inputenc}
\usepackage[T1]{fontenc}
\usepackage{hyperref}
\usepackage{graphicx}
\usepackage{pgf}
\usepackage{url}
\usepackage{fancyhdr}

\renewcommand{\labelitemi}{$-$} %usar '-' como bullet
\setlength{\parskip}{1ex}
\setlength{\parindent}{0pt}

\begin{document}
\title{\textbf{simon} wireframes and implementation guide}
\author{\small{André Santos}\\\small{Márcio Coelho}}
\pagestyle{fancy}
\date{\today}
\maketitle
\begin{abstract}
This document contains wireframes to help as a visual guide in the development of the core and interface components of Simon. When appropriate, brief comments allow to define more specific details.
\end{abstract}
%\addvspace{2cm}
\tableofcontents

%\section{Introduction}
\section{Public Access}
\subsection{Homepage}
\pgfimage[width=\textwidth]{wf/homepage}
The homepage includes four main elements: the logo, the sidebar, the annoucements and the blog.
\begin{description}
\item[Logo]: At this point, this is simply the Simon logo (later it should be possible to change it to a customized logo -- file upload feature needed for this).
\item[Sidebar]: This will contain links to Static Pages. However, in Settings there will be a variable containing the HTML for this sidebar, which should be possible to modify. A micro-DSL should be defined to ease the editing of the content of the sidebar.
\item[Annoucements]: A banner which displays a slideshow of the active announcements. For reference, see \url{http://www.tmn.pt}.
\item[Blog]: The main section of the page. Should behave like a standard blog, displaying the last posts.
\end{description}

\subsection{View Post}
\pgfimage[width=\textwidth]{wf/homepage_post_view}
As a normal user (signed in or not), this should be how a post is viewed. If the user is signed in, the add comment box should appear at the bottom.

\section{User Management}
\subsection{Dashboard}

\subsection{Users}
\subsubsection{Sign Up}
\pgfimage[width=\textwidth]{wf/user_signup}
There are two different ways of signing up: either with or without Open ID. Both ways should be clearly separated.
\subsubsection{Sign In}
\pgfimage[width=\textwidth]{wf/user_signin}
There are two different ways of signing in: either with or without Open ID. Both ways should be clearly separated.

\subsubsection{List}
\pgfimage[width=\textwidth]{wf/user_list}
\subsubsection{View}
There are two different View User cases: an user viewing its own profile (or and admin viewing an user profile), or an user viewing other user's profile.
\paragraph{As self or Admin}~\\

\pgfimage[width=\textwidth]{wf/user_view_self}
\paragraph{As other normal user}~\\

\pgfimage[width=\textwidth]{wf/user_view_other}

\subsubsection{Edit}
There are two different Edit User cases: an user editing its own profile, and an admin editing anyone's profile. These are presented by growing order of editable information:
\paragraph{As self}~\\

\pgfimage[width=\textwidth]{wf/user_edit_self}
\paragraph{As admin}~\\

\pgfimage[width=\textwidth]{wf/user_edit_admin}

\subsection{Groups}
\subsubsection{New}
\pgfimage[width=\textwidth]{wf/group_new}
\subsubsection{List}
\pgfimage[width=\textwidth]{wf/group_list}
\subsubsection{View}
\pgfimage[width=\textwidth]{wf/group_view}
\subsubsection{Edit}
\pgfimage[width=\textwidth]{wf/group_edit}

\section{Website}
\subsection{Announcements}
\subsubsection{New}
See \textbf{Edit Announcement}.
%\pgfimage[width=\textwidth]{wf/announcement_new}
\subsubsection{List}
\pgfimage[width=\textwidth]{wf/announcement_list}
\subsubsection{View}
\pgfimage[width=\textwidth]{wf/announcement_view}
\subsubsection{Edit}
\pgfimage[width=\textwidth]{wf/announcement_edit}
\subsubsection{Remove}

\subsection{Posts}
\subsubsection{New}
\pgfimage[width=\textwidth]{wf/post_new}
\subsubsection{List}
\pgfimage[width=\textwidth]{wf/post_list}
\subsubsection{View}
\pgfimage[width=\textwidth]{wf/post_view}
\subsubsection{Edit}
\pgfimage[width=\textwidth]{wf/post_edit}

\subsection{Comments}
\subsubsection{New}
See the \textbf{View Post} wireframe.
%\pgfimage[width=\textwidth]{wf/comment_new}
\subsubsection{List}
\pgfimage[width=\textwidth]{wf/comment_list}
\subsubsection{View}
See the \textbf{View Post} wireframe.
%\pgfimage[width=\textwidth]{wf/comment_view}
\subsubsection{Edit}
\pgfimage[width=\textwidth]{wf/comment_edit}
Only the admin will have the power to edit existing comments.

\subsection{Tags}
\pgfimage[width=\textwidth]{wf/tag_list}
\begin{itemize}
\item A table with the tags list should be visible on the Post listing page.
\item Clicking on a tag will filter the Posts list, displaying only the posts marked with that tag.
\item Due to the \textbf{Remove Comment} feature, it is possible to have tags which are used in \textbf{zero} posts. This tags should not be visible on the tags list.
\item A maximum number N of tags to show should be defined. Only N or less tags should be displayed on the tag list.
\end{itemize}

\subsection{Static Pages}
\subsubsection{New}
\pgfimage[width=\textwidth]{wf/page_new}
\subsubsection{List}
\pgfimage[width=\textwidth]{wf/page_list}
\subsubsection{View}
\pgfimage[width=\textwidth]{wf/page_view}
\subsubsection{Edit}
\pgfimage[width=\textwidth]{wf/page_edit}

\section{System}
\subsection{Dashboard}
%TODO 
Aqui devem aparecer estatisticas ou assim
\subsection{Settings}
\subsubsection{List}
\subsubsection{View}
\subsubsection{Edit}
\subsubsection{Remove}

\subsection{Actions}
\subsubsection{List}
\subsubsection{View}
\subsubsection{Edit}
\subsubsection{Remove}


%\subsection{Users}
%\subsection{Users}
%\subsection{Users}
%\subsection{Users}
%\subsection{Users}
%\subsection{Users}
% Um cronograma relativo às várias tarefas pode ser consultado na Figura~\ref{fig:crono}.
% \begin{figure}[h!]
% \centering
% \includegraphics[width=.8\textwidth]{crono}
% \caption{Cronograma das várias tarefas a realizar no âmbito desta dissertação.}
% \label{fig:crono}
% \end{figure}

\end{document}
